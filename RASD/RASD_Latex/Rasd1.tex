	\chapter{Introduction}
		\section{Purpose}
			\subsection{General purpose}
				\paragraph{}
					SafeStreets is a crowd-sourced application that wants to provide users with the possibility to notify authorities when parking violations occur. 
					
					The application will allow users to send pictures of violations, including their date, time, and position. Then authorities will be able to check, validate and eventually use those data for issuing tickets.
					
					 To achieve this objective many goals have been identified, this vital aims will have to be met before the release of the product.
				\paragraph{}
					In short, the S2B will satisfy the following goals:
					 \begin{itemize}
					 	\item {[G1]} The system must allow logged-in users to send a report of the violation
					 	\item {[G2]} The system must allow logged-in users to see their past reports
					 	\item {[G3]} The system must allow logged-in users to retrieve information about the position and types of valid reports
					 	\item {[G4]} The system must allow verified authorities to mine information about date, time, position and type of valid reports
					 	\item {[G5]} The system must allow verified authorities to retrieve statistics about valid reports
					 	\item {[G6]} The system must be able to cross the data retrieved from the municipality with its own, in order to identify unsafe areas and suggest possible interventions to the municipal employee
					 	\item {[G7]} The system must allow local officer to set the validity of a report sent by the user
					 	\item {[G8]} The system must ensure that the chain of custody of the information coming from the user to the municipality is never broken, and the information is never altered
					 \end{itemize}
					 
			\paragraph{}
				W.r.t. {[G6]} an area is considered "unsafe" if and only if a minimum number of problems is met on that area.
			\subsection{Document purpose}
				\paragraph{}
					This document represents the Requirement Analysis and Specification Document (RASD). This document aims to completely describe the system in terms of functional and non-functional requirements, analyze the real needs of the users to model the system, show the constraints and the limit of the software and indicate the typical use cases that will occur after the release. 
					
					This document is addressed to the developers who have to implement the requirements and could be used as a contractual basis.
		\section{Scope}
			\paragraph{}
				This service is born from the idea that social responsibility on the street can be achieved with the help of everyday citizens. Such an objective is achievable by giving good-willed people the possibility to record parking violations that they spot on the street and making them visible to the authorities later. To this kind of people, regarded as unregistered users before their subscription to the service, the choice of signing up is given.
				
			\paragraph{}
				When an unregistered user signs up, he/she will become a registered user, able to log in whenever desired. A logged-in registered user, to employ the functionalities of the system, must have at least a mobile phone with a camera and a GPS localization system, otherwise, the product won't be available for use. With the minimum requirements satisfied the user will be able to compile and send reports of the parking violations and, if interested, search in a selected area for violations that will be shown as dots on an interactive map provided by the map service.
 
			\paragraph{}
				Each report will be composed by the type of the violation, i.e. "vehicle parked in a forbidden area", by a picture of the vehicle with its license plate highlighted, that will be later recognized and added to the report with an OCR service, by the date, hour and position where the picture has been taken. Registered users will also be able to see their past reports that, if the authority has already judged them as genuine, will be recognized as valid and highlighted with a green check. Invalid reports will be recognizable by a red cross and reports which evaluation is still pending will be represented by a yellow clock.
			\paragraph{}
				The authorities, embodied by the municipal employees and local officers, will be able to retrieve data from the system using any available device capable of connecting to the internet and running a browser. In particular, both the municipal employees and the local officers will be able to extract the reports sent by the users as one or more reports, choosing time, date, area or type of violation. If a group of reports has been chosen they will be showed in the same kind of way as a user, like dots on a map, but more information will be made available. Furthermore, both the municipal employees and the local officers will be able to retrieve statistics, derived from data collected by the system. The system data is collected with the registered users' reports, along with the ticket and accident information possessed by the municipality. If the authority possesses such data, it will be retrieved via a ticket service, that will fetch date, time, position and violation type of the vehicle found committing the infraction, and a municipal accident service, that will recover information about road accidents, such as the date, time and the vehicles involved in the accident. Moreover the municipal employee, by selecting a position, will be able to search for possible improvements on such position, to see what kind of interventions should be made, and, if some of them have already been completed, he/she can change their status from "not done" to "done", a "done" improvement won't be shown again.
			\paragraph{}
				The last functionality is dedicated to the local officers: after having withdrawn a report, the local officer is able to check its validity, and eventually utilize its data to fine the vehicle that committed the violation.

		\section{Definitions, Acronyms, Abbreviations}
			\subsection{Definitions}
				\begin{itemize}
					\item \textbf{Report:} Collection of Data that represents a Violation, in particular
						\begin{itemize}
							\item Picture: a photo of the vehicle that has been found committing a violation
							\item Date: the date when the picture has been taken
							\item Time: the hour when the picture has been taken
							\item Position: the place, formatted using GPS location, of the vehicle that has been found committing a violation.
						\end{itemize}
					\item \textbf{Improvement:} a possible road intervention finalized to the development of the road and to achieve a safer environment
				\end{itemize}
			\subsection{Acronyms}
				\begin{itemize}
					\item UU = Unregistered User;
					\item RU = Registered User;
					\item ME = Municipal Employee;
					\item LO = Local Officer;
					\item S2B = Software to Be;
					\item TS = Ticket Service;
					\item MS = Map Service;
					\item MAS = Municipal Accident Service;
					\item OCRS = OCR Service;
					\item VT = Violation type;
				\end{itemize}
				\paragraph{}
					For a precise description of RU, ME and LO see section \ref{sez:UserCharacteristics}
					
			\subsection{Abbreviations}
				\begin{itemize}
					\item {[Gn]} : n-th goal.
					\item {[Dn]} : n-th domain assumption
					\item {[Rn]} : n-th functional requirement
				\end{itemize}
		\section{Revision history}
			\begin{itemize}
				\item Version 1.0
					\begin{itemize}
						\item Initial release
					\end{itemize}
				\item Version 1.1
					\begin{itemize}
						\item Changed document appearance
						\item Fixed names and coherence between chapters
						\item Partial revision of use cases and scenarios
						\item Fixed alloy model and description
						\item Partial revision of the domain assumptions and requirements w.r.t. the authorities' credentials
						\item Expansion of the security performance section
						\item Correction of loops in sequence diagrams
						\item Correction of useless requirements about issued tickets (old R20)
						\item Correction of report validation, with valid report sent to the ticket service
					\end{itemize}
			\end{itemize}
		\section{Reference documents}
			\begin{itemize}
				\item \textit{Specification document:} ''Mandatory Project Assignment AY 2019/2020''
			\end{itemize}
		\section{Document structure}
			\paragraph{}
				The RASD document is composed by six chapters, as outlined below:
			\begin{description}
				\item[Chapter 1] describes the purpose of the system and the list of goals that the application has to reach. Moreover, it defines the scope, where the aim of the project is defined and the application domain with the shared phenomena are shown.
				\item[Chapter 2] offers an overall description of the project. Here the actors, involved in the application's usage, are identified and the boundaries of the project are defined, listing all the necessary assumptions. Moreover, a class diagram is provided, to better understand the general structure of the project. Then some state diagrams are listed to make the evolution of the crucial objects and actors clear. Finally, the functions offered by the system are here more clearly specified, with respect to the previously listed goals.
				\item[Chapter 3] represents the body of the document. It contains the interface requirements, which are: user interfaces, hardware interfaces and software interfaces. It then lists some scenarios to show how the system acts in real-world situations, followed by the description of the functional requirements, using use cases and sequence diagrams. All the requirements necessary to reach the goals are given, linked with the related domain assumptions. Lastly, the non-functional requirements are defined through performance requirements, design constraints and software system attributes.
				\item[Chapter 4] contains the Alloy model of some primary aspects with all the related comments and documentation in order to show how the project has been modeled and represented through the language.
				\item[Chapter 5] shows the effort which each member of the group has spent working on the project
				\item[Chapter 6] contains eventual references used during the writing of the document
			\end{description}